\documentclass{article}
\usepackage[utf8]{inputenc}
\usepackage{amsmath}
\usepackage{amsfonts}
\usepackage{mathrsfs}
\usepackage{graphicx}
\usepackage{subcaption}
\usepackage{hyperref}
\usepackage[top=20truemm,bottom=20truemm,left=20truemm,right=20truemm]{geometry}

\title{Class Project Description}
\author{Kansuke Ikehara (Kansuke.Ikehara@colorado.edu)}

\begin{document}
\maketitle
Stochastic Block Model (SBM) is a generative model used for generating random networks with some structures, including community structure, core-periphery structure, etc. \cite{SBM}. In this model, each node (vertex) is assigned to one of communities. Edges are placed between nodes with the probability $p$ which depends on the nodes' membership to the communities. For example, nodes in the same group, say $c_{1}$ could more likely be connected than those which both are in different groups, $c_{i}$ and $c_{j}$. One of the advantages of SBM is that we could control the structure of a network with a handful parameters. In this class project, I am going to construct a temporal SBM model which takes a set of three parameters , which evolve according to some chaotic equation, such as Lorenz equation. The idea is that those three parameters which control the network structure evolve over time as a state-space trajectory in a chaotic equation. I then construct networks based on the resulting parameters and investigate some properties of them.

The Following are a more detailed description of the parameters (they are still very abstract, meaning that I will have to ponder on them more): $C$, a quantity measuring the density of the entire network (mean degree could be useful),  $C_{in} - C_{out}$, a measure of how skewed connections are between communities where $C_{in}$ and $C_{out}$  are edge densities within a community and between communities, respectively , and $\gamma$, the ratio of communities' size. In this model, we assume there only exists two communities for the sake of simplicity.
\bibliographystyle{ieeetr}
\bibliography{reference} 
\end{document}












